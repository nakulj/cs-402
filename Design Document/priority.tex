\newcommand{\bpr}{\code{base_priority}\ }
\newcommand{\dpr}{\code{donated_priority}\ }
\newcommand{\epr}{\code{effective_priority}\ }
\newcommand{\ta}{\code{a}\ }
\newcommand{\tb }{\code{b}\ }

\subsection{Data Structures}
	\subsubsection{}
		\begin{itemize}

\item \code{struct thread} \begin{itemize}
		\item \code{int base_priority}\\The base priority of each thread.
		\item \code{int donated_priority}\\The highest priority any thread is donating to each thread.
		\item \code{int effective_priority}\\The higher of the two above priorities.
\end{itemize}

\item \code{struct semaphore} \begin{itemize}
	\item \code{struct list waiters}\\List of threads waiting on the semaphore.
\end{itemize}

\end{itemize}


	\subsubsection{}
		Priority donations are tracked using the \dpr field of the \code{struct thread}. Because the field is calculated to be up-to-date every time locks are released, it is ensured that \dpr represents the highest donated priority to the field. 
		
\subsection{Algorithms}
	\subsubsection{}
		In the \code{lock_sema_up()} function (which is called as a part of the \code{lock_release()}) function), we recalculate the \epr of all threads waiting on that lock. Then, the current thread is forced to yield, calling the scheduler, whose \code{next_thread_to_run()} method ensures that only the highest priority thread on the \code{ready_list} is selected to run.
	\subsubsection{}
			Every thread is initialised with a certain \bpr and a \dpr of zero. Any time either of these values is changed, the \epr is recalculated to be the higher of the \bpr  and \dpr  values.
		When a thread \ta attempts to acquire a lock held by thread \tb, \ta donates its effective priority to \tb (\code{b->donated_priority= a->effective_priority}), which then recursively donates the priority to any threads that might be holding locks that \tb is waiting on.
		Because threads donate their \epr rather than their \bpr, nested donation is automatically tracked.
	\subsubsection{}
		Whenever locks are released, the \dpr and \epr of all affected threads are recalculated. This allows for arbitrary levels of nesting.
\subsection{Synchronization}
	\subsubsection{}
		A potential race hazard arises if, after determining to assign \bpr to \code{effective_priority}, another thread donates a higher value to \code{donated_priority}. This would cause a miscalculation of the \code{effective_priority}. This is avoided by disabling interrupts during the \code{set_priority()} operation.
		
		We cannot use a lock to solve this issue as it would involve donating priorities, which could possibly cause deadlocks.
\subsection{Rationale}
	\subsubsection{}
		We chose this design due to the simplicity of implementing it. We had also considered using a stack to track nested priority donations; however, this would have placed arbitrary limits on levels of donation nesting. Further, it would complicate handling the case in which a donor thread's base priority was changed while it was waiting for a lock.